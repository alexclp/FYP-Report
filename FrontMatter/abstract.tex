The recent developments in smart-phone technologies, such as Wi-Fi, have made it possible to build very accurate indoor positioning systems. Moreover, mobile processors have advance capabilities which allow them to render high definition augmented reality objects. 

As such, the propose of this project is to develop an indoor positioning system that uses a fingerprinting approach to Wi-Fi positioning, in order to provide positioning and navigation to students in the Department of Informatics. Additionally, the project aims to provide a better navigation experience for users, by adding augmented reality visual cues, and provide information for nearby reference points, i.e. timetables for computer labs.

Using the proposed method, a localisation accuracy of $<$5 meters is achieved. The system is also able to find a path in under 2-3 seconds, and show augmented reality objects. To incorporate relevant information for some of these objects, data is supplied by the King's College London timetable service and PC-Free@King's. 

The system has an extensible design, separated in a multitude of independent subsystems. In the future, this means that subsystems can be easily replaced in order to enable use on a variety of platforms and devices. Furthermore, with simple adjustments, the system can be adapted for use in any building, given the availability of floor plans. Therefore, there is an opportunity to continue the project further to encompass all King's College London campuses, providing navigation to all students and staff.