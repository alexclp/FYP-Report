The recent developments in smartphone technology, such as Wi-Fi technologies have made it possible to build very accurate indoor positioning systems. Moreover, the recent development of powerful mobile chips has facilitated advancements in areas such as augmented reality.

This paper will propose an indoor positioning system that uses a fingerprinting approach to map Bush House, one of King's College London buildings, using the available floor plans for it. Additionally, the project aims to provide a better navigation experience for users, by adding augmented reality visual cues, and provide information for nearby reference points, i.e. computer labs.

Using the proposed localisation method, an accuracy of under 5 meters is achieved. The system is able to find a path in under 2-3 seconds, and show augmented reality objects built with information from King's College London timetable service and PC-Free@King's. Finally, the system has an extensible design, separated in a multitude of independent subsystems, that can be easily replaced, regardless of the platform. Furthermore, the design of the indoor positioning system can be used for any building, given the available floor plans, if small adjustments are made.