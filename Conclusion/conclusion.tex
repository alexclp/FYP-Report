The project proves the reliability and feasibility of developing an IPS, given that the floor plans are available. It shows that it is possible to develop a very precise and robust positioning and navigation system to determine indoor locations of Wi-Fi enabled devices, by only using the existing infrastructure available. Therefore, no additional cost is incurred, and no dedicated positioning devices are required, such as GPS chips.

The project aimed to develop an IPS that is different way to previous solutions. Firstly, the IPS was built using a fingerprinting method, combined with existing floor plans. Secondly, the IPS was developed on iOS, which is rare due to the number of limitations imposed by Apple. Although not perfect, the project has been successful in terms of achieving the aims that have been set at the beginning of project. 

There is a lot of new research in the field of indoor positioning, which means that in the future many accurate IPSes can be produced. In particular, in this project, an indoor navigation system with built-in AR features has been studied and developed. Therefore, as more research is conducted in this area, improvements can be made to the app, which will hopefully improve accuracy, making it suitable for use by a wider audience and in multiple use cases.